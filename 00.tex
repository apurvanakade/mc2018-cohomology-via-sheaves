% !TEX root = index.tex
\maketitle
\tableofcontents
\newpage

\setcounter{section}{-1}
\section{Introduction \& Motivation}
\epigraph{In mathematics you don't understand things. You just get used to them.}{John von Neumann}

Cohomology was introduced by Poincare in a series of papers named \emph{Analysis Situs} and now forms the basis of modern Algebraic Topology.
To a topological space $X$ we can associate a sequence of vector spaces denoted
\begin{align*}
    \check H^i(X)
\end{align*}
for each $i \in \Z_{\ge 0}$ called it's \textbf{Cech Cohomology} (pronounced \emph{check cohomology}). In a very loose sense, the dimension of  $\check H^i(X)$ measures the $i^{th}$ dimensional holes in $X$.\\

Why care about the $i^{th}$ dimensional holes? We can use these to rigorously distinguish between spaces. For example, most proofs of the fact that $\R^m$ is not homeomorphic\footnote{Homeomorphism is the isomorphism for topological spaces.} to $\R^n$ if $m \neq n$ use some cohomology computation. We'll also see that a torus $S^1 \times S^1$ has two \emph{1-dimensional holes} which distinguishes it from a sphere $S^2$ which has none.

\begin{thm}
  Two topological spaces $X$, $Y$ are homeomorphic only if
  \begin{align*}
    \check H^i(X) \cong \check H^i(Y)
  \end{align*} for all non-negative integers $i$.
\end{thm}
\begin{remark}
  The above statement is not an \emph{if and only if} statement. The other direction is easily shown to be false. You'll be able to come up with examples by yourself in a couple of days.
\end{remark}

Computing the cohomology requires multiple steps. The goal of this class is to develop the relevant machinery and actually do some cohomology computations.
\begin{align*}
 \xymatrix@R-2pc{
 X \ar@{~>}[r] & \U \ar@{~>}[r] & \L^\bullet(\U) \ar@{~>}[r] & \check H^*(X)\\
 \mbox{Topological Space} & \mbox{Good cover of } X & \mbox{Cech Complex of } \U & \mbox{Cech Cohomology of } X
 }
\end{align*}

\noindent \textbf{Note:} The number of stars (*) on the problems indicate their difficulty level. The non-starred marked problems are compulsory, the starred problems are optional.

% We'll gradually define Cech cohomology. The strategy is the following.
% \begin{enumerate}
% 	\item For a space $ X$, find a good open cover $ \U$
% 	\item Find the Cech complex (to be defined) $ \L^\bullet(\U)$ associated $ \U$
% 	\item Find the cohomology of the complex $H^*(\L^\bullet(\U))$ which then equals the singular cohomology of the space $H^*(X, \F)$.
% \end{enumerate}
%
% \begin{remark}
% 	When $X$ is a nice enough space (which includes most spaces you can think of) the Cech cohomology $\check H^i(X; \F)$ equals another purely topological invariant of $X$ called the \textbf{singular homology}. And thus Cech cohomology provides a purely algebraic way of computing a topological invariant.
% \end{remark}
