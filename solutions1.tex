\input{preamble}
\usepackage{scrextend}


\usepackage{pgfplots}
\usepgfplotslibrary{polar}
\pgfplotsset{compat=newest}

% \includeonly{02}

\title{Cohomology via Sheaves: Solutions to Selected Solutions}
\author{\small{Apurva Nakade}}
\date{}
\fancyhead[LO]{Cohomology via Sheaves: Solutions to Selected Problems}

\begin{document}

\section{Topological Preliminaries: Solutions}
\vspace{3em}
\noindent \textbf{Question 4. g)} Good cover for $S^2$:\\
\begin{addmargin}[2em]{0em}
For any triangulation of the sphere $S^2$, the faces of the triangulation form a good cover. This is more generally true for any surface.
\begin{figure}[H]
		\centering
		\includegraphics[height=5cm]{SphereTetrahedron}
    \caption{Triangulating a Sphere. (googled image)}
	\end{figure}
  \vspace{6em}
\end{addmargin}


\noindent \textbf{Question 5.} Good cover for $X \vee Y$:\\
\begin{addmargin}[2em]{0em}
In general, if $X$ and $Y$ have good covers $\U$ and $\U'$ such that the `point of gluing' belongs to a unique $U \in \U$ and $U' \in \U'$. Then we can just use the open covers $\U$ and $\U'$ but glue the open sets $U$ and $U'$ at a point. We can easily check that this indeed forms a good cover. The only non-trivial fact needed to be shown is that if $U$ and $U'$ are contractible then so is $U \vee U'$ (exercise).
\end{addmargin}


\newpage

\noindent \textbf{Question 2. b)} If $X \times Y$ is contractible then so is $X$:\\
\begin{addmargin}[2em]{0em}
(I think you don't need the contractibility of $Y$, if you find a flaw in the following proof please let me know.)

Because $X \times Y$ is contractible, there is a point $(x_0, y_0) \in X \times Y$ and a continuous map
\begin{align*}
  \Phi: X \times Y \times [0,1] \rightarrow X \times Y
\end{align*}
such that
\begin{align*}
  \Phi(x,y,0) & = (x,y)   \\
  \Phi(x,y,1) & = (x_0,y_0)
\end{align*}
for all $ (x,y) \in X \times Y$ i.e. there are ``continuously varying paths'' connecting each point in $ X \times Y$ to $ (x_0,y_0)$.

Restrict the map $\Phi$ to $X \times \{ y_0 \}$ so that we get a map
\begin{align*}
  \Phi|_{y=y_0}: X \times [0,1] &\rightarrow X \times Y \\
  (x,t) &\mapsto \Phi(x,y_0,t)
\end{align*}
This is almost the map we want. Now we compose with the projection onto the first component $\pi_X: X \times Y \rightarrow X$ to get a map $\Psi = \pi_X \circ \Phi$
\begin{align*}
  \Psi : X \times [0,1] \xrightarrow{\Phi|_{y=y_0}} X \times Y \xrightarrow{\pi_X} X
\end{align*}
One can check that $\Psi(x,0) = x $ and $\Psi(x,1) = x_0$. Further, $\Psi$ is continuous as it is a composition of two continuous functions and hence $\Psi$ witnesses the contractibility of $X$.

(Try to draw pictures of what this proof is saying geometrically.)
\end{addmargin}

\end{document}
